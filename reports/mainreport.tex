\documentclass[parskip=full]{scrreprt}

\usepackage[english]{babel}
\usepackage[utf8]{inputenc}
\usepackage{csquotes}
\usepackage[backend=biber]{biblatex}
\addbibresource{mainreport.bib}

\usepackage{graphicx}
  \graphicspath{ {./graphics/} }
\usepackage{url}
\usepackage{varioref}
\usepackage{tabularx}
  \newcolumntype{L}{>{\raggedright\arraybackslash}X}
\usepackage[version=4]{mhchem}
  
\author{Arin Wongprommoon\\University of Cambridge}
\title{Optimising production of citramalate based on the \emph{E. coli} kinetic model}
\subtitle{Project Report}
\date{17 August 2018}

\begin{document}

\maketitle

\tableofcontents

\begin{abstract}
  Using living organisms to synthesise chemicals is an alternative to synthesising chemicals from fossil fuels, as it serves as a renewable resource with potentially high efficiency and low cost. The procedures can be sped up by optimising conditions using mathematical models. Citramalate ((2S)-2-Hydroxy-2-methylbutanedioate) is a chemical of industrial interest as it can be a precursor for methacrylic acid, a monomer for the production of plastics. 
  
  The project aimed to find conditions to maximise the production of citramalate using modelling approaches. In this project, an existing kinetic model for \emph{E. coli} metabolism was extended by adding a reaction that produces citramalate from acetyl coenzyme A and pyruvate. \emph{E. coli} was chosen as the model organism as its metabolism is very well characterised in the literature. The project employed Python and libraries specific to genetic algorithms, modelling, and manipulating information presented in SBML (systems biology markup language).
  
  First, the effect of $V_{max}$ values of enzymes in the kinetic model~\cite{millard_metabolic_2017} on productivity of citramalate was investigated. $V_{max}$ values were chosen as a parameter as it can be easily tested \emph{in vivo} by relying on the principle that $V_{max}$ is proportional to enzyme concentration. The differential evolution genetic algorthim was then employed to compute the set of $V_{max}$ values of enzymes that optimises the production of citramalate, assuming Michaelis-Menten kinetics.
  
  In the second part of the project, information from the kinetic model was used to enrich the stoichiometric model~\cite{orth_comprehensive_2011}. More specifically, information about the possible values of fluxes through each reaciton was used to set the boundaries for each reaction in a stoichiometric model. Flux balance analysis (FBA) was then performed to evaluate the highest possible flux through the citramalate-producing reaction as a proxy for citramalate productivity. The composition of the Lund medium~\cite{eastham_process_2015} was studied in an attempt to create boundaries for relevant uptake reactions.
  
  The project took place at the Cambridge Systems Biology Centre, Department of Biochemistry, in Prof Steve Oliver's group. Research associate Dr Jorge J\'ulvez supervised me throughout the project. The project was entirely computational, and ran from 25 June 2018 to 17 August 2018.
\end{abstract}

\chapter*{Introduction}
\label{ch:intro}

The project concerns two models of \emph{E. coli} metabolism: a kinetic model described by Millard \emph{et al} in 2017~\cite{millard_metabolic_2017} and a stoichiometric model described by Orth \emph{et al} in 2011~\cite{orth_comprehensive_2011}. The kinetic model concerns a smaller set of reactions than the stoichiometric model, but it contains information about initial conditions (substrate concentrations and flux through reactions) which the stoichiometric model lacks.

The kinetic model contains 68 reactions, 49 of which include $V_{max}$ as a parameter. Of these, 41\footnote{ACEA, ACEB, ACK, ACN\_1, ACN\_2, ACS, ATP\_syn, CITRA\_SYN, CYTBO, EDA, EDD, ENO, FBA, FBP, FUMA, GDH, GLT, GND, GPM, LPD, MAD, MDH, MQO, PCK, PDH, PFK, PGI, PGK, PGL, PIT, PPC, PPS, PTA, PYK, RPE, RPI, SDH, SK, SQR, TPI, and ZWF} correspond to real enzyme-catalyzed reactions in \emph{E. coli}. To investigate citramalate production, a 69th reaction called CITRA\_SYN was added to the model. It models the reaction:

\begin{center}
  acetyl-CoA + pyruvate + \ce{H2O} $\rightarrow$ CoA-SH + \ce{H^+} + citramalate
\end{center}

with the kinetic law

\[
  \frac{\mathrm{d}[citramalate]}{\mathrm{d}t} = 
  \frac{V_{max} \cdot [acetyl-CoA]}{[acetyl-CoA] + K_{m}}
\]

assuming pyruvate is saturating. The $V_{max}$ is set to 4 mM s\textsuperscript{-1} and $K_{m}$ to 0.495 mM in the modified model.

Citramalate productivity is defined as $\mu Y_{P/S}$, where $\mu$ is the growth rate in reciprocal time units (h\textsuperscript{-1} in this project) and $Y_{P/S}$ is defined as mass of product over mass of substrate, which is glucose in this context. Manipulating the SBML file required the Python library \texttt{libsbml} and running simulations employed the \texttt{roadrunner} module. Throughout the project, the simulations were from 0 to 7200 seconds, at which steady-state is attained for almost all the conditions (also see section~\vref{sec:couples}) %change this to a subsection investigating steady-state later

The stoichiometric model contains 2,584 reactions, and I have mapped 57 of them to their equivalents in the kinetic model (more details in section~\vref{sec:mapping}). By default, most of these reactions are unbounded (Orth \emph{et al}~\cite{orth_comprehensive_2011} provided details about this). In a similar vein to the kinetic model, a 2,585th reaction for citramalate production was added to the stoichiometric model using the COBRA module in Python. However, no kinetic parameters were added, as stoichiometric models do not contain this information. In addition to the citramalate synthesis reaction, a sink reaction to allow citramalate to leave the system was also added.

\chapter{Investigating the kinetic model}
\label{ch:kinetic}

%(Maybe write the purpose of doing this here? But I think the abstract and the introduction suffice)

\section{Varying $V_{max}$ values of one reaction at a time}
\label{sec:onereac}

Here I investigated the relationship between the value of $V_{max}$ of each reaction in the kinetic model and the resulting citramalate productivity at steady state -- i.e.\ after two hours. I varied the $V_{max}$ values of each of the 49 reactions\footnote{ACEA, ACEB, ACK, ACN\_1, ACN\_2, ACS, ATP\_MAINTENANCE, ATP\_syn, CITRA\_SYN, CYTBO, EDA, EDD, ENO, FBA, FBP, FUMA, GDH, GLT, GND, GPM, GROWTH, LPD, MAD, MDH, MQO, NADH\_req, NDHII, PCK, PDH, PFK, PGI, PGK, PGL, PIT, PPC, PPS, PTA, PYK, RPE, RPI, SDH, SK, SQR, TPI, XCH\_ACE1, XCH\_ACE2, XCH\_GLC, XCH\_ZWF} in the kinetic model that include $V_{max}$ as a parameter. The ranges 0.1--1.0 $V_{max}$ and 0.1--10.0 $V_{max}$, where $V_{max}$ is the wild-type $V_{max}$ for that particular reaction as specified in the SBML file. I used 100 data points for each plot. Figure~\ref{fig:onereacsample} is an example of such a plot.

\begin{figure}[htbp]
  \centering
  \includegraphics[scale=0.5]{onereacsample}
  \caption{Example of a one-reaction plot: the reaction PYK}
  \label{fig:onereacsample}
\end{figure}

For some enzymes, varying $V_{max}$ values have a greater effect on citramalate productivity than others. I quantified this effect based on the data points of the 0.5 -- 2.0 $V_{max}$ plots\footnote{created early in the project, not included in the final results. I only mention them here because I used information from them in later parts of the project} as follows:

\[
effect = \frac{(max - min)}{(0.5 \cdot (max + min)}
\]

This is the difference between the maximum and minimum productivities obtained over the range, relative to the average, which represents where the value of productivity is. I intended to find the extent of variation from the `typical' value of productivity around this range. Granted, this isn’t the best way to analyse. I’ve tried regression, but it didn’t seem very informative.

This identified CITRA\_SYN, GLT, LPD, GROWTH, ATP\_MAINTENANCE, GDH, ATP\_SYN, ACEA, PYK, and ZWF as among the enzymes that had the greatest effects. Millard \emph{et al}~\cite{millard_metabolic_2017} listed CYTBO, ZWF, GDH, GLT, and GROWTH as the enzymes found to have the largest shares of flux control. They also exert the strongest controls on concentration. These enzymes are followed by LPD, ATP\_MAINTENANCE, ACEA, ATP\_SYN, and PYK

Exchange reactions (XCH\_ACE, XCH\_GLC, and XCH\_P) have no bearing on productivity as all the values in their plots can be attributed to noise. Surprisingly, ACS (acetyl CoA synthetase), ACK (acetate kinase), and PTA (phosphate acetyltransferase), which are all directly related to controlling acetyl CoA levels, do not seem to have ignificant bearings on productivity.

As expected, enzymes found to have the largest shares of flux control create the plots that demonstrate the greatest change in citramalate productivity in response to $V_{max}$ changes. However, CYTBO has a less effect on productivity as would be expected by this explanation. Although PYK does not exert a lot of control over fluxes as expected by its flux control coefficient, it has quite a great effect on citramalate productivity. Possibly it is because the citramalate reaction uses up pyruvate and because PYK is a control point in glycolysis. A few reactions exhibit inflection points in their plots, as shown in table~\ref{tab:inflection}. For reference, the wild-type productivity is 0.001122 h\textsuperscript{-1}. %0.00112225918623

\begin{table}[htbp]
  \caption{Reactions with inflection points}
  \label{tab:inflection}
  \centering
  \begin{tabular}{lrrl}
    Reaction & $V_{max}$ (mM s\textsuperscript{-1}) & Productivity (h\textsuperscript{-1}) & Type\\
    ATP\_syn & 16.804 & 0.00216 & minimum\\
    & 23.724 & 0.00231 & maximum\\
    CITRA\_SYN & 0.545 & 0.01660 & maximum\\
    CYTBO & 5.124 & 0.00113 & maximum\\
    EDA & 0.014 & 0.00112 & maximum\\
    GDH & 4.569 & 0.00149 & maximum\\
    MQO & 1.597 & 0.00098 & minimum\\
    PFK & 0.145 & 0.00113 & maximum
  \end{tabular}
\end{table}

In addition to the list of reactions ordered by effect on citramalate productivity as described earlier, I also extracted a list of reactions arranged by descending order of flux control coefficent (FCC) from the Millard \emph{et al} paper, and used it alongside this list in later parts of the project.

\section{Varying $V_{max}$ values of two reactions at a time}
\label{sec:couples}

Continuing on from the one-reaction exercises, this part aims to vary the $V_{max}$ values of pairs of reactions to investigate their effect on citramalate productivity. Ideally, I should have looked at all the possible pairs from the 49 reactions that have $V_{max}$ as a parameter -- i.e. 1,176 of them -- but I have limited time, so I choose eight reactions off the top of the list of ‘enzymes’ that caused the greatest relative change in productivity with changing $V_{max}$ (0.5 to 2.0) as found earlier.

These are: CITRA\_SYN, GLT, LPD, ATP\_MAINTENANCE, GDH, ATP\_syn, ACEA, ZWF.

With these I varied their $V_{max}$ values from 0.1 to 10.0 times their wild-type $V_{max}$ values, and plotted the resulting citramalate productivities on heatmaps. Twenty-eight heatmaps were generated.

\subsection{Results}
\label{ssec:couplesresults}

The heatmaps are in the relevant folder - I don’t feel like attaching 28 images in this report.

(Add one sample `normal' heatmap here)

Plots that include ATP\_MAINTENANCE as one of the reactions exhibited strange behaviour, which becomes more evident when the resolution of the plots are increased:

(Add the weird ATP\_MAINTENANCE heatmap here)

Generating high-resolution one-reaction plots (citramalate productivity against $V_{max}$) revealed the problem:

(Add the weird ATP\_MAINTENANCE 1D plot here)

Jorge suggests that it’s because the model is not equipped to deal with small values of $V_{max}$.

Apart from that, the trends seen in all the other plots are exactly as expected if you mesh two one-reaction plots together. In other words, the effects are additive without any additional local maxima or minima in the spaces of the plots.

\subsection{Synergistic effects}
\label{ssec:synergistic}

Whether there are synergistic effects is difficult to determine. I’ve investigated enzymes adjacent to each other. No special effects were observed for PGI and PFK, and some slight synergistic effect was observed for GPM and ENO. Between FBA and GDH, FBA exerts too little effect for the results to be conclusive. In contrast, ACEA and ACEB exhibit \emph{antagonistic} effects, and LPD and GDH, a pair of enzymes far apart from each other in the network seem to have a synergistic effect.

\subsection{Steady-state checking}
\label{ssec:steadystate}

For each simulation I looked at the rate of change of the concentration of each species in the system at the end of the simulation, two hours (7,200 seconds). I then took the rate with the greatest absolute value, and compared it with a threshold value. Initially, the threshold value was 1e-8. If the greatest absolute value is greater than the threshold value, then the simulation is taken to have not reached steady state. I then incorporated this information into heatmaps. (Obviously I used the computer to do all of these steps for me, but I’m writing it this way to avoid making my explanation convoluted.)

(Add SS plot here)

Additionally, I produced heatmaps showing the greatest absolute values themselves (on a log scale).

With this, I noticed that:\\
ATP, ADP, P, and Hout are among the species that break the 1e-8 threshold, however, they stay at around 1e-7\\
ATP\_MAINTENANCE at 0.1 $V_{max}$ produced greatest absolute values on the order of 1e-1\\
ATP\_syn on the order of 1e-2\\
For these two, the species that caused the problem are BPG and OAA\\
GDH on the order of 1e-3, and here the species that caused the problem are GLCx and GLCp

After a discussion with Jorge, a threshold of 1e-6 is reasonable, and it is safe to say that almost all reactions achieve steady state, apart from situations where ATP\_MAINTENANCE, ATP\_syn, and GDH have very low $V_{max}$ values.

% This is redundant, will deal with it later

Removing these species reveals:\\
BPG for all reactions involving ATP\_MAINTENANCE, at around 0.09\\
GLCx/GLCp for all reactions involving GDH, at the order of 0.000X\\
FDP for CITRA\_SYN vs GDH; GLT vs GDH\\
SUC for GLT vs LPD\\
OAA for ATP\_MAINTENANCE vs ATP\_syn\\
SUCCOA for GDH vs ATP\_syn\\
G6P for ATP\_syn vs ZWF

Looking at the heatmaps for the max concentration rates:\\
As expected, ATP\_MAINTENANCE reliably screws everything up in its low $V_{max}$ ranges, giving rates at the order of 1e-1 while every other data point in the heatmap ranges from 1e-6 to 1e-11\\
But so does ATP\_syn (order of 1e-2 to 1e-3 generally) in its low $V_{max}$ ranges. ATP\_MAINTENANCE and ATP\_syn are obviously responsible for the ATP derivates listed earlier.\\
GDH produces 1e-3 values in its low $V_{max}$ ranges - this is the GLCx/GLCp\\
Apart from that, no obvious patterns

\section{Using differential evolution}
\label{sec:de}

Differential evolution is a genetic algorithm. I’ve used it to maximise the citramalate productivity while multiple enzymes in a list are varied in their $V_{max}$ values at the same time. Strictly speaking, the algorithm minimises the value of a function, but with the use of one minus sign, I got it to maximise what I need.

Differential evolution has the following parameters:
F\\
CR\\
NP\\
D (dimensions)-- is the number of enzymes varied at once\\
number of iterations/generations

2. Using differential evolution to validate couples heatmaps

In general agreement -- sometimes discovers maxima not seen by the heatmaps.

\subsection{Finding optimal parameters}
\label{ssec:deoptimise}

I tested the values quoted by Storn and MEH Pedersen (2010) on optimising citramalate productivity by playing with enzymes in the glycolytic pathway (why it is the glycolytic pathway will be evident later). Values are chosen based on the times it takes for the algorithm, how fast the algorithm exhibits convergence, and how well the algorithm returns roughly the same values each time it is run.

(Add table here)

\subsection{Glycolytic enzymes}
\label{ssec:glycolytic}

Enzymes: PGI, PFK, FBA, GDH, PGK, GPM, ENO, PYK, PDH

Originally I used differential evolution with this set of glycolytic enzymes in order to investigate whether the $V_{max}$ values of all enzymes in a pathway have to be increased to get an appreciable increase in the overall flux through the pathway, resulting in higher productivity. Niederberger et al (1992) stated in their paper that flux through all enzymes in a linear pathway must be increased for flux through the pathway to be increased, judging by their experiments on the tryptophan synthesis pathway in yeast. However, Yamamoto et al (2012) stated otherwise, and showed that appreciable increases resulted after increasing the flux through only one or two enzymes. My results support the latter.

The objective of this exercise changed to investigating how much adding an enzyme as an additional dimension in the differential evolution algorithm affects productivity.

Adding enzymes relatively low on the minmax.ods list does not affect the optimum $V_{max}$ values already on there much, and does not cause big leaps in productivity. This is nice because we can get away with not looking at all enzymes in a pathway to optimise productivity, producing a good solution for the memory leak (the corners we have to cut because the writers of RoadRunner didn't deal with this \ldots). The `curse of dimensionality' ruins everything - even if there's no memory leak, I will have to deal with running the damn thing for an awfully long time. So it was worth finding out with enzymes had the most effects on productivity after all.

\subsection{Optimisation of citramalate production}
\label{ssec:optcitra}

Used the best parameters on differential evolution on the set of seven enzymes: CITRA\_SYN, GLT, LPD, GDH, ATP\_syn, ACEA, ZWF

F = 0.6607, CR = 0.9426, NP = 28, generations = 50

(Insert table for 7 citra enzymes here)

Productivity = 353.66 10-4

Confirmed that DE is useful: took values expected to generate the highest productivity as expected from one-reaction plots alone and plugged them into the model, then used comproducti() to evaluate productivity. Turns out to be 226.96. Conclusion: DE is useful.

Tried it on 10 enzymes - CITRA\_SYN, GLT, LPD, GDH, ATP\_syn, ACEA, PYK, ZWF, NDHII, MQO -- achieved 405.5 productivity, but values swing.

(Insert table for 10 citra enzymes from the old de report here)

\chapter{Enriching the stoichiometric model}
\label{ch:stoich}

An Anargyros who used to work with this group mapped reactions between the kinetic model described by Millard et al (2017) and the stoichiometric model by Orth et al (2011). He took into account reactions that were reversed with respect to each other and differing stoichiometries. He produced a mapping table, sub-network diagrams, and lower/upper limits for fluxes of each reaction common to both models. The focus was more on mapping and he didn’t put specific constraints while generating the lower/upper limits, and the method by which he generated these values was not documented.

Flux balance analysis is employed, and aims to find an optimal solution for an objective reaction (e.g.\ flux of citramalate synthesis reaction) subject to constraints to the possible flux values. Essentially this is linear programming. Orth et al (2010) wrote a good explanation (add reference here)

Reconciling unit differences

In the kinetic model, units of flux through each reaction are in mM s-1, while it is mmol gDW-1 h-1 in the stoichiometric model. Taking into account the cell volume of 1.77e-3 L gDW-1as specified in the kinetic model, I’ve worked out that to convert the units from the kinetic model to the stoichiometric model, multiply by 6.372.

Because of this relationship and the fact that the stoichiometric model does not have any information of enzyme kinetics, I can safely use numbers with kinetic model units on the structure of the stoichiometric network, and if necessary, apply the unit conversion at any time (or never at all)

\section{Mapping reactions in the kinetic model to the stoichiometric model}
\label{sec:mapping}

Steal stuff from the mapping report. Mostly repeat that. Not sure if I intended the main report to have more or less detail than the mapping report -- my memory is horseshit.

\section{Creating boundaries for FBA}
\label{sec:bounds}

Aim: finding the minimum and maximum of the fluxes through the 68 reactions in the kinetic model (69 for the version with citramalate added) by varying the $V_{max}$ values of the 49 enzymes with $V_{max}$ as a parameter.

Ideally: vary all 49 at once, but computational memory and time is limited, so I have devised two strategies to produce educated guesses for the lower and upper limits of the flux values. The broader the boundaries, the better they are.

A) Varying one enzyme at a time

B) Varying many enzymes at a time using differential evolution to obtain minimum and maximum values.

First method: varying one enzyme at a time. Found by varying the $V_{max}$ values of one enzyme at a time over the range of 0.1-10.0 $V_{max}$ in a modification of the Millard et al (2017) kinetic model, namely, with the citramalate reaction added

Varying many enzymes (will just not mention 4D, 5D, 6D, 7D as its unnecessary)

8D: Use original list. CYTBO, MQO, MDH, ZWF, GLT, GDH, ATP\_syn, ACK. Over range of 0.3 -- 10.0 $V_{max}$ because of CVODE. Say something about $V_{max}$ not reliable in this region anyway. State DE parameters. Maybe put the boundaries in the appendix.

41D: List of all enzymes. Over 0.4 -- 10.0 because CVODE. State DE parameters.

\section{Flux balance analysis}
\label{sec:fba}

Add results from CitraFluxResults.txt here -- easier just to write it \emph{de novo} instead of Frankensteninig sentences from different report files/Evernote.

Then, instead of using the flux through the citramalate reaction as the objective function in the FBA, I looped through all 2,585 reactions in the model and used each as an objective function, for both the minimising and maximising senses. The results are (of course) in large CSV files.

First I used the boundaries that were included in the stoichiometric model (i.e. no constraints), then applied the 41D boundaries. The optimal minimum and maximum values will be used to create a further set of constraints for each reaction in the stoichiometric model later.

\section{Investigating glucose uptake}
\label{sec:glucoseuptake}

For reference, this is a mapping between the relevant reactions that I settled on.

(Insert table showing mapping here)

First, I looked at the relationship between the glucose feed and the fluxes through the other reactions in the kinetic model. The fluxes through XCH\_GLC and the PTS\_X reactions are always the same. XCH\_GLC is equal to GLC\_feed from GLC\_feed = 0 to GLC\_feed = 0.68, after which XCH\_GLC never increases above 0.68 (saturation).

With increasing GLC\_feed, GLCx and GLCp concentrations remain very low until GLC\_feed = 0.68, after which they increase linearly as GLC\_feed increases. The concentration of G6P increases linearly until GLC\_feed = 0.68, at which it immediately plateaus at 3.19

\section{Using information from the Lund medium}
\label{sec:lund}

As specified on page 79 of the patent WO 2015/022496, a medium for E. coli BW25113 $\Delta{}pflB\Delta{}ldhA$ transformed with pBAD24-cimA fermentation for the optimal production of (R)-citramalic acid has the composition:

Insert composition in table form here

I calculated the ratios between the moles of each species present in this recipe (will make it into a more presentable format later):

With these ratios, I applied them as bounds (strictly lower bounds, but they represent the maximum rate of uptake because of sign conventions used by the model) for the relevant exchange reactions in the stoichiometric model, fixing glucose feed bound to -0.23, -0.68, and -10.0. The model has exchange reactions for all the above species except for EDTA, and I merged the phosphoric acid derivatives into one ‘phosphates’ species for the model.

Only the glucose and citrate exchange reaction appears in the flux values in the optimal solution. Upon totally unbounding all these exchange reactions (some are bound to 0 in the default model), glucose exchange is -11.13 and citrate exchange is -1.20

\chapter{Remarks}
\label{ch:remarks}

Donec at rutrum tellus, ut posuere urna. Nulla facilisi. Quisque ut orci tincidunt, ultricies eros sed, ultrices purus. Praesent scelerisque nunc urna, in pharetra augue pharetra ut. In quis quam consectetur, finibus orci sit amet, molestie massa. Nullam vel dictum lorem. 

\section{Issues in the project}
\label{sec:issues}

Memory leak, things that don't make sense, etc. Things that get in the way of me making revolutionary discoveries.

Vestibulum sed est molestie, feugiat justo vitae, aliquam augue. Ut accumsan massa dignissim, maximus nulla at, imperdiet ligula. Donec vitae tortor vitae libero dapibus maximus. Fusce a enim faucibus, sollicitudin sapien nec, maximus nibh. Morbi pharetra egestas lorem, non semper arcu faucibus vitae. Morbi vehicula urna ac odio blandit gravida. Integer id magna pharetra, gravida odio et, pulvinar magna. Praesent consequat elementum mollis.

\section{Future direction and suggestions}
\label{sec:future}

Nunc orci justo, ultricies id pharetra a, consequat at dolor. Curabitur posuere nisl eget leo viverra, non cursus nunc consectetur. Fusce non turpis non sem porttitor cursus id a lacus. Mauris iaculis sit amet risus ut viverra. Nunc eget arcu magna. Donec quis suscipit metus. Donec aliquet imperdiet consectetur. Ut tempus, dui et sodales posuere, diam est sagittis est, nec consectetur mi dolor nec mi. Ut in lacus vitae sapien lacinia aliquam in tincidunt nisi. Etiam non pretium justo. Phasellus augue lectus, condimentum vel orci et, suscipit semper dolor. Fusce at finibus nulla.

\section{Notes about files}
\label{sec:files}

Notes about the scripts, text files, image files, spreadsheets etc.\ used. Notes about Git/GitHub.

Sed nec erat iaculis lectus rhoncus ornare. Duis ante ex, sodales sed mi sed, blandit venenatis dui. Donec vel ornare odio, laoreet viverra felis. Phasellus aliquam mi et sollicitudin fermentum. Ut vitae mollis lacus. Sed tempor nulla at rutrum iaculis. Vestibulum convallis elementum nibh, fermentum porttitor leo vulputate eget. Sed vel sagittis elit, eu faucibus ante. Aliquam erat volutpat. Donec mattis sit amet ante quis luctus. Fusce lectus risus, molestie at tincidunt non, consectetur nec massa. Duis arcu justo, lacinia ut nulla tristique, congue interdum purus. Maecenas finibus, urna eu mollis commodo, nisi metus tincidunt velit, at convallis libero justo id turpis. Etiam iaculis sodales quam vitae fringilla. 

\chapter*{Appendix}
\label{ch:appendix}

%\section

% Here I intend to put lists or figures that would be annoying if in the main text

\nocite{*}
\printbibliography

\end{document}
