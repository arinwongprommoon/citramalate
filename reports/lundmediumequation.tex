\documentclass[a4paper]{scrartcl}

% language
\usepackage[english]{babel}
\usepackage[utf8]{inputenc}
\usepackage{csquotes}
\usepackage{url}
\usepackage[version=4]{mhchem}
\usepackage{siunitx}
\DeclareSIUnit\molar{\mole\per\cubic\deci\metre}
\DeclareSIUnit\Molar{\textsc{m}}
\DeclareSIUnit\calorie{cal}
\usepackage{physics}

% citations
\usepackage[
  natbib=true,
  backend=biber,
  doi=true,
  isbn=false,
  url=false,
  date=year,
  style=authoryear,
  citestyle=authoryear,
  minnames=1,
  maxnames=2,
  minbibnames=1,
  maxbibnames=99]{biblatex}
\addbibresource{lundmediumequation.bib}
\usepackage{varioref}

% floats
\usepackage{graphicx}
\graphicspath{ {./graphics/} }
\usepackage[font=small]{caption}
\usepackage[font=small]{subcaption}

\title{Adressing the Lund medium}
\date{}

\begin{document}

\maketitle

Previously I devised a method to
use the concentrations of the species present in the Lund medium to set the bounds of exchange reactions.
This method is not derived from any existing methodology in the literature; I just did what made sense to me and resulted in consistent units.
This method uses the following information:

\begin{itemize}
\item a concentration of a chemical species $[X_{i}]$
\item the growth rate or dilution rate $\mu$
\item a measured OD\textsubscript{600}, and
\item that an OD\textsubscript{600} of 1 corresponds to an \emph{E. coli} concentration of $Z$, expressed in g\textsubscript{DW} L\textsuperscript{-1} (0.36--0.39 g\textsubscript{DW} L\textsuperscript{-1} in the literature, but \citet{nanchen_nonlinear_2006} quotes 0.41 g\textsubscript{DW} per unit OD\textsubscript{600})
\end{itemize}
  
the maximum exchange rate $R$ for that chemical species can be calculated as follows:

\[
  R = \frac{\mu[X_{i}]}{OD_{600} \cdot Z}
\]

And thus the corresponding reaction in the stoichiometric model will have the lower flux boundary of $-R$ and the upper flux boundary of 0.

\citet{orth_comprehensive_2011} describe experimental phenotypic screens for essentiality of a 1,074-gene subset of iAF1260 genes. The conditions included glucose M9 minimal medium under aerobic conditions in 96-well plates. The OD\textsubscript{600} values of the screens from this conditions roughly conform to a normal distribution, with a mean OD\textsubscript{600} value of 0.335.
Using $\mu = 0.1$ (as in the kinetic model), an OD\textsubscript{600} value of 0.335, and $Z = 0.36$, maximum exchange rates for species present in the Lund medium were calculated.

However, using these values results in a very low calculated maximum glucose uptake rate along with much lower rates for other chemicals present in the Lund medium.
The Lund medium has glucose at a concentration of \SI{11.9}{\gram\per\litre}, and using the values stated gives a maximum uptake rate $R$ of \SI{0.00238}{\milli\Molar\per\second}. The models may not be tuned for such a low value.

It is possible that such a low value may be because I used an $OD_{600}$ value derived from phenotype screening experiments by \citet{orth_comprehensive_2011}, which use very different conditions compared to the continuous culture the kinetic model by \citet{millard_metabolic_2017} assumes.

\citet{chassagnole_dynamic_2002} concern the development of a dynamics model that links the sugar transport system of \emph{E. coli} with glycolysis and the pentose phosphate pathway.  Their study utilises growth conditions consistent with the model by \citet{millard_metabolic_2017}: \SI{37}{\celsius}, pH 7.0, steady-state conditions, and a dilution rate ($D$) of \SI{0.1}{\per\hour}.  The medium used includes \SI{20.0}{\gram\per\litre} glucose.

The study presents a model concerning glucose uptake:

\[
  \dv{C_{glc}^{ext}}{t} = D(C_{glc}^{feed} - C_{glc}^{ext}) + f_{pulse} - \frac{C_{x}r_{PTS}}{\rho_{x}}
\]

Where:

\begin{itemize}
\item $D$ is the dilution rate, \SI{0.1}{\per\hour} in this study, equal to that of the kinetic model
\item $C_{glc}^{feed}$ is the concentration of glucose in the medium fed to the continuous culture, \SI{20.0}{\gram\per\litre} in this study
\item $C_{glc}^{ext}$ is the external concentration of glucose in the chemostat, measured to be \SI{0.0556}{\milli\Molar} in this study
\item $f_{pulse}$ is a variable that accounts for the sudden change of glucose concentration caused by the glucose pulse experiments in the study
\item $C_{x}$ is the biomass concentration, measured to be 8.7 g\textsubscript{DW} bacteria per litre of continuous culture
\item $r_{PTS}$ is the rate of glucose uptake; and
  \item $\rho_{x}$ is the specific weight of biomass (i.e. the density of the biomass), measured to be 564 g\textsubscript{DW} L\textsuperscript{-1}.
  \end{itemize}

  Substituting $\dv{C_{glc}^{ext}}{t} = 0$ for steady-state conditions, setting $f_{pulse}$ to be zero (assuming no glucose pulse experiments are undertaken; these experiments are not relevant to our kinetic model), and setting other variables as above gives:

  \[
    r_{PTS} = \frac{\rho_{x}D}{C_{x}}(C_{glc}^{feed} - C_{glc}^{ext}) = \SI{0.200}{\milli\Molar\per\second}
  \]

  This is on the same order as the \SI{0.23}{\milli\Molar\per\second} default glucose feed rate for the kinetic model.  \citet{nanchen_nonlinear_2006} reports that the total cellular carbon influx varies linearly with dilution rate, thus validating how $r_{PTS} \propto D$ in this equation.

  To obtain the maximum uptake rate for glucose, I think it is reasonable to set $C_{glc}^{ext}$ to zero in the above expression to represent a situation in which the glucose uptake is increased by so much that it eliminates all the extracellular glucose -- glucose uptake cannot possibly be greater than this.  Thus this gives:

  \[
    r_{PTS}^{max} = \frac{\rho_{x}D}{C_{x}}C_{glc}^{feed}
  \]

  If the concentration of glucose in the Lund medium is used in this expression, with all other variables remaining the same, the expression will give a maximum uptake rate of about \SI{0.12}{\milli\Molar\per\second}.  \ce{Zn^2+} has the lowest molarity in the Lund medium; the maximum \ce{Zn^2+} uptake rate would thus be about \SI{3.6e-6}{\milli\Molar\per\second}.

  However, I think method may have some caveats.  I can imagine $\rho_{x}$ being near-constant among different chemostat cultures with the same dilution rate as it looks like an physical property of bacteria (i.e. density).  But, $C_{x}$ seems like a value specific to a certain growth condition used in a chemostat, and may depend on the glucose concentration used.  \citet{chassagnole_dynamic_2002} use a medium with \SI{20.0}{\gram\per\litre} and report a $C_{x}$ of 8.7 g\textsubscript{DW} L\textsuperscript{-1}, while \citet{taymaz-nikerel_development_2009} and \cite{taymaz-nikerel_escherichia_2011} use a medium with \SI{30.0}{\gram\per\litre} and report $C_{x}$ values within 10 $\sim$ 11 g\textsubscript{DW} L\textsuperscript{-1}.  This calls into question how applicable these equations apply to using information from the Lund medium to enrich the stoichiometric model.

\printbibliography

\end{document}